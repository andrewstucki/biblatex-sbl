% imakeidx needs to be loaded before hyperref
\RequirePackage{scrlfile}\AfterPackage{textcomp}{\RequirePackage{imakeidx}}
\documentclass[a4paper]{ltxdockit}[2011/03/25]
\usepackage{microtype}
\usepackage{xcolor}
\makeindex[title=Author Index,intoc,options=-q]
\usepackage{btxdockit}
\usepackage{fontspec}
\usepackage{realscripts}
\usepackage{xparse}
\usepackage{framed}
\usepackage{enumitem}

\usepackage[style=sbl,indexing=cite,idemtracker=false,ibidtracker=false]{biblatex}
\addbibresource{biblatex-sbl.bib}

\hypersetup{colorlinks,citecolor=spot}

\hyphenation{Prei-sen-daz}

\setmonofont{DejaVu Sans Mono}[Scale=MatchLowercase]
\setromanfont{Linux Libertine O}
\setsansfont{Linux Biolinum O}[
  BoldItalicFont={* Bold},
  BoldItalicFeatures={FakeSlant=0.2}
]

\newcommand*{\biblatexsbl}{\sty{biblatex-sbl}\xspace}
\newcommand*{\biblatexsblhome}{https://github.com/dcpurton/biblatex-sbl/}
\newcommand*{\biblatex}{\sty{biblatex}\xspace}

\ExplSyntaxOn
\NewDocumentCommand \samplemacro { m }
  {
    \texttt{#1}\par
  }
\NewDocumentCommand \sblrefsamplecite { s m m m o o m }
  {
    \IfNoValueTF { #5 }
      {
        \IfNoValueT { #6 }
          {
            \IfBooleanF { #1 }
              {
                \samplemacro{\textbackslash #2\{#7\}}
              }
            \hspace*{\bibindent}#4\csuse{#3}{#7}
          }
      }
      {
        \IfNoValueTF { #6 }
          {
            \IfBooleanF { #1 }
              {
                \samplemacro{\textbackslash #2[#5]\{#7\}}
              }
            \hspace*{\bibindent}#4\csuse{#3}[#5]{#7}
          }
          {
            \IfBooleanF { #1 }
              {
                \samplemacro{\textbackslash #2[#5][#6]\{#7\}}
              }
            \hspace*{\bibindent}#4\csuse{#3}[#5][#6]{#7}
          }
      }
  }
\NewDocumentCommand \samplecite { s m o o m }
  {
    \rmfamily
    \IfBooleanTF { #1 }
      {
        \sblrefsamplecite*{autocite}{cite}{#2.~}[#3][#4]{#5}.\par
      }
      {
        \sblrefsamplecite{autocite}{cite}{#2.~}[#3][#4]{#5}.\par
      }
  }
\NewDocumentCommand \sampleparencite { s o o m }
  {
    \rmfamily
    \IfBooleanTF { #1 }
      {
        \sblrefsamplecite*{parencite}{parencite}{}[#2][#3]{#4}\par
      }
      {
        \sblrefsamplecite{parencite}{parencite}{}[#2][#3]{#4}\par
      }
  }
\NewDocumentCommand \samplebib { s m }
  {
    \IfBooleanF { #1 }
      {
        \samplemacro{\textbackslash printbibliography}
      }
    \hangindent\bibindent\bibentrycite{#2}.\par
  }
\NewDocumentCommand \samplebiblist { s m }
  {
    \IfBooleanF { #1 }
      {
        \samplemacro{\textbackslash printbiblist\{abbreviations\}}
      }
    \biblistcite{#2}
  }
\ExplSyntaxOff

\lstset{%
  basicstyle=\displayverbfont\normalsize,
  keywordstyle=\bfseries
}

\makeatletter
\def\ltd@printarg@v<(#1)#2>{[(\prm{#1})\prm{#2}]\ltd@parseargs}
\makeatother

\titlepage{%
  title={\biblatexsbl},
  subtitle={SBL Style Using \biblatex},
  url={\biblatexsblhome},
  author={David Purton},
  email={dcpurton@marshwiggle.net},
  revision={\printsblversion},
  date={\printsbldate}}

\hypersetup{%
  pdftitle={biblatex-sbl},
  pdfsubject={SBL Style Using biblatex},
  pdfauthor={David Purton},
  pdfkeywords={sbl, biblatex, bibliography, citation}}

\definecolor{spot}{rgb}{0.25,0.25,0.65}
\colorlet{shadecolor}{black!15}

\begin{document}

\printtitlepage

\tableofcontents

\section{Introduction}

\biblatexsbl provides support to \biblatex and LaTeX for citations,
bibliography, and a list of abbreviations in the style recommended by the
Society of Biblical Literature (\citeshorthand{SBL}). The style conforms to
the second edition of the \cite{SBLHS}.

The style supports all examples given in the handbook (see
\sty{biblatex-sbl-examples.pdf}). Shorthand citations and a list of
abbreviations containing journals, series, and shorthands are handled
automatically. Repeated authors in the bibliography are replaced by a
horizontal line. \emph{Ibidem} and \emph{idem} is enabled by default. Indexing
of names is supported, but not enabled by default. Only note style citations,
not Author-Date citations are supported. Primary sources can be cited in
parentheses. \biblatexsbl is compatible with \biblatex's support for
\sty{hyperref}.

For anything not covered in this manual, please see the \biblatex
documentation. Bugs and feature requests can be submitted at
\url{\biblatexsblhome}.

\textbf{Note:} This package should be considered as beta software and its
output carefully checked when you use it.

\section{Requirements}

\biblatexsbl requires at least version 3.11 of \biblatex and the \sty{xparse}
package. \sty{biber} must be used. \sty{bibtex} is not supported. For
localization \sty{babel} (not \sty{polyglossia}) and \sty{csquotes} are
recommended.

\section{Usage}

The following minimal example will set up \biblatexsbl to conform to the
defaults of the \cite{SBLHS}.

\begin{quote}
\begin{lstlisting}[style=latex]{}
\documentclass{article}
\usepackage[style=sbl]{biblatex}
\addbibresource{<bibfile.bib>}
\begin{document}
\printbiblist{abbreviations}
\printbibliography
\end{document}
\end{lstlisting}
\end{quote}

\subsection{Localization}

By default \biblatexsbl uses American style punctuation and quotation marks.
You can choose a different style by including the
\sty{babel} and \sty{csquotes} packages in your
document preamble. \sty{polyglossia} is not well supported by \biblatex and
its use is discouraged.

\begin{quote}
\begin{lstlisting}[style=latex]{}
\usepackage[ngerman]{babel}
\usepackage{csquotes}
\usepackage[style=sbl]{biblatex}
\end{lstlisting}
\end{quote}

Currently \opt{english} (including variants such as \opt{british},
\opt{australian}, etc.), \opt{spanish}, and \opt{german} are supported.

For Greek and especially Hebrew, the set up is more complicated and you should
use \sty{xelatex} or \sty{lualatex}. See \sty{sbl-paper.pdf} for example Greek
and Hebrew usage with \sty{babel}.

\begin{quote}
\begin{lstlisting}[style=latex]{}
\usepackage[nil,bidi=default]{babel}
\usepackage{csquotes}
\babelprovide[import=en-US,main]{american}
\babelprovide[import=he]{hebrew}
\babelprovide[import=el]{polutonikogreek}
\babelfont[american]{rm}[Ligatures=TeX]{Linux Libertine O}
\babelfont[hebrew]{rm}%
  [Ligatures=TeX,Contextuals=Alternate]{SBL BibLit}
\babelfont[polutonikogreek]{rm}%
  [Ligatures=TeX,Contextuals=Alternate]{SBL BibLit}
\end{lstlisting}
\end{quote}

\subsection{Commands}

The standard commands for \biblatexsbl generally follow those defined by
\biblatex. Included below are the most typical commands. For more commands and
options, reference the \biblatex manual.

\begin{ltxsyntax}

\cmditem{autocite}[prenote]<(altpostnote)postnote>{key}

\cmd{autocite} inserts a citation as a footnote. If used in a footnote, the
citation is placed in parentheses. It works as in the standard \biblatex
styles, except that that \bibfield{postnote} argument can be divided into two
using parentheses. This creates an \bibfield{altpostnote} field which is used
in some of the examples from §6.4 of the \cite{SBLHS}. e.g.,

\begin{snugshade}
  \samplecite{1}[See][(1.3)8:223]{clementinehomilies}
\end{snugshade}

To use only \bibfield{altpostnote} surround the whole argument in parentheses.
e.g.,

\begin{snugshade}
  \samplecite{1}[(III. 1-164)]{PGM:betz}
\end{snugshade}

\cmditem{cite}[prenote]<(altpostnote)postnote>{key}

\cmd{cite} works in the same way as \cmd{autocite} except that the citation is
placed directly into the text instead of in a footnote.

\cmditem{parencite}[prenote]<(altpostnote)postnote>{key}

\cmd{parencite} works in the same way as \cmd{autocite} except that the
citation is placed inside parentheses instead of in a footnote. This is most
useful for citing primary sources. e.g.,

\begin{snugshade}
  \sampleparencite[2.233-235]{josephus:ant}
\end{snugshade}

\cmditem{journalcite}{key}
\cmditem{seriescite}{key}
\cmditem{shorthandcite}{key}

\cmd{journalcite}, \cmd{seriescite}, and \cmd{shorthandcite} inserts the
respective abbreviation into the text and also adds it to the list of
abbreviations. The abbreviation is hyperlinked to the list of abbreviations if
the \sty{hyperref} package is loaded.

These commands ignore the \bibfield{prenote} and \bibfield{postnote} fields,
so can safely be used anywhere within a database entry.

\cmditem{printbiblist}

This command prints a bibliography list. In \biblatexsbl all abbreviations
(journals, series, and shorthands) can be printed using the following command:

\begin{quote}
  \verb+\printbiblist[...]{abbreviations}+
\end{quote}

See the \biblatex manual for an explanation of available optional arguments.

\cmditem{printbibliography}

Inserts the bibliography. See the \biblatex manual for an explanation of
available optional arguments.

\end{ltxsyntax}

\subsection{Package Options}

\biblatexsbl defaults to the recommendations of the \citeshorthand{SBL}, but
it also supports many of the standard options from \biblatex as well as a few
custom ones outlined below.

\begin{optionlist}

\optitem[false]{accessdate}{\opt{true}, \opt{false}}

The \cite{SBLHS} discourages the use of access
dates.\autocite[See][§6.1.6, 84]{SBLHS} If they are required this option can be
passed to \biblatex.

\optitem[sbl]{citepages}{\opt{sbl}, \opt{permit}, \opt{omit}, \opt{separate}}

Use this option to fine-tune the formatting of the \bibfield{pages} field
the first time an entry is cited.

\begin{valuelist}
\item[sbl] The \bibfield{postnote} field is not printed for first citations.
  e.g.,

  \begin{snugshade}
    \samplecite{1}[159]{leyerle:1993}
  \end{snugshade}

  If \bibfield{postnote} is not a page range, then it is printed in
  parentheses after \bibfield{pages}. e.g.,

  \begin{snugshade}
    \samplecite{1}[a note]{irvine:2014}
  \end{snugshade}

  The one exception to this is the \bibtype{incommentary} entry type which
  always sets \opt{citepages} to \opt{omit} (see below) when \bibfield{volume}
  is defined.\autocite[See][§1.3.3.2]{SBLHS:studentsupp}

\item[permit] The \bibfield{postnote} is printed in parentheses after the
  \bibfield{pages} field. e.g.,

  \makeatletter\cbx@opt@citepages@permit\makeatother

  \begin{snugshade}
    \samplecite{1}[245]{wildberger:1965}
  \end{snugshade}

\item[omit] The \bibfield{pages} field is not printed unless
  \bibfield{postnote} is empty or not a page range (in which case behaviour
  matches \opt{citepages=sbl}). e.g.,

  \makeatletter\cbx@opt@citepages@omit\makeatother

  \begin{snugshade}
    \samplecite{1}[5]{freedman:1977}
  \end{snugshade}

\item[separate] The \bibfield{postnote} is printed in parentheses after the
  \bibfield{pages} field preceeded by the bibliography string \sty{thiscite}.
  e.g.,

  \makeatletter\cbx@opt@citepages@separate\makeatother

  \begin{snugshade}
    \samplecite{1}[1]{petersen:1988}
  \end{snugshade}

  If \bibfield{postnote} is not a page range, then \sty{firstcite} is not
  used and the behaviour matches \opt{citepages=sbl}.

  \makeatletter\cbx@opt@citepages@sbl\makeatother

\end{valuelist}

\boolitem[true]{dashed}

By default, this style replaces recurrent authors/editors in the bibliography
by a dash so that items by the same author or editor are visually grouped.
This feature is controlled by the package option \opt{dashed}. Setting
\opt{dashed=false} in the preamble will disable this feature.

\optitem[comp]{eprintdate}{\opt{year}, \opt{short}, \opt{long}, \opt{terse,
\opt{comp}, \opt{iso8601}}}

Similar to the \opt{date} option (for details see the \biblatex manual) but
controls the format of the \bibfield{eprintdate}.

\boolitem[false]{fullbibrefs}

The \emph{Student Supplement for the} \cite{SBLHS} permits two styles for the
bibliography entry for Bible dictionaries, encyclopaedias, and multivolume
commentaries for the entire Bible by multiple
authors.\autocite[4–5]{SBLHS:studentsupp}

This option applies to \bibtype{inreference} and \bibtype{incommentary}
entry types.

\begin{valuelist}
\item[true] The bibliography entry is printed in long form. e.g.,

  \begin{snugshade}
    \toggletrue{fullbibrefs}
    \nocite{IDB}
    \samplebib*{stendahl:1962}
    \togglefalse{fullbibrefs}
  \end{snugshade}

\item[false] The bibliography entry is printed in a short form. e.g.,

  \begin{snugshade}
    \samplebib*{stendahl:1962}
  \end{snugshade}
\end{valuelist}

\optitem[constrict]{ibidtracker}{\opt{true}, \opt{false}, \opt{context},
\opt{strict}, \opt{constrict}}

This option controls the \emph{ibidem} tracker. The possible choices are:

\begin{valuelist}
\item[true] Enable the tracker in global mode.
  not tracked separately between text body and footnotes.
\item[false] Disable the tracker: \emph{ibid.}\ will not be used.
\item[context] Enable the tracker in context-sensitive mode. In this mode,
  citations in footnotes and in the body text are tracked separately.
\item[strict] Enable the tracker in strict mode. In this mode, potentially
  ambiguous references are suppressed. A reference is considered ambiguous if
  either the current citation (the one including the \emph{ibidem}) or the
  previous citation (the one the \emph{ibidem} refers to) consists of a list
  of references.
\item[constrict] This mode combines the features of \opt{context} and
  \opt{strict}. It also keeps track of footnote numbers and detects
  potentially ambiguous references in footnotes in a stricter way than the
  \opt{strict} option. In addition to the conditions imposed by the
  \opt{strict} option, a reference in a footnote will only be considered as
  unambiguous if the current citation and the previous citation are given in
  the same footnote or in immediately consecutive footnotes.
\end{valuelist}

\boolitem[false]{ibidpage}

The scholarly abbreviation \emph{ibidem} is sometimes taken to mean both ‘same
author + same title’ and ‘same author + same title + same page’ in traditional
citation schemes. By default, this is not the case with this style because it
may lead to ambiguous citations. If you prefer the wider interpretation of
\emph{ibidem}, set the package option \opt{ibidpage=true} or simply
\opt{ibidpage} in the preamble. The default setting is \opt{ibidpage=false}.

\optitem[constrict]{idemtracker}{\opt{true}, \opt{false}, \opt{context},
\opt{strict}, \opt{constrict}}

This option controls the \emph{idem} tracker. The possible choices are:

\begin{valuelist}
\item[true] Enable the tracker in global mode.
\item[false] Disable the tracker: \emph{idem} will not be used.
\item[context] Enable the tracker in context-sensitive mode. In this mode,
  citations in footnotes and in the body text are tracked separately.
\item[strict] This is an alias for \opt{true}, provided only for consistency
  with the other trackers. Since \emph{idem} replacements do not get ambiguous
  in the same way as \emph{ibidem}, the strict tracking mode does not apply to
  them.
\item[constrict] This mode is similar to \opt{context} with one additional
  condition: a reference in a footnote will only be considered as unambiguous
  if the current citation and the previous citation are given in the same
  footnote.
\end{valuelist}

\optitem[true]{pagetracker}{\opt{true}, \opt{false}}

This option controls whether \emph{ibidem} and \emph{idem} are used across
page breaks or not.

\begin{valuelist}
\item[true] Enable the tracker in automatic mode. This is like \opt{spread} if
  LaTeX is in twoside mode, and like \opt{page} otherwise.
\item[false] Disable the tracker.
\item[page] Enable the tracker in page mode. In this mode, tracking works on a
  per-page basis.
\item[spread] Enable the tracker in spread mode. In this mode, tracking works
  on a per-spread (double page) basis.
\end{valuelist}

\boolitem[true]{sblfootnotes}

This option controls the style of footnotes. This option is compatible with
the \sty{footmisc} package provided \sty{footmisc} is loaded before \biblatex.

\begin{valuelist}
\item[true] Footnotes are printed with a normal number followed by a period
  and the first line indented:

  \begin{snugshade}
    \samplecite*{1}{talbert:1992}
  \end{snugshade}

\item[false] Footnotes are printed with a superscript (or whatever other
  default has been set up by your style):

  \begin{snugshade}
    \hspace*{\bibindent}\llap{\textsuperscript{1}}\cite{robinson+koester:1971}.
  \end{snugshade}
\end{valuelist}

\optitem[true]{shorthand}{\opt{true}, \opt{false}, \opt{short}, \opt{intro}}

This option controls when and whether the \bibfield{shorthand} field is used
as a citation. This can also be used as a type option or entry option.
\bibtype{ancienttext} and \bibtype{classictext} entry types ignore this
option.

\begin{valuelist}
\item[true] Always use the \bibfield{shorthand} when citing the entry.
\item[false] Never use the \bibfield{shorthand} when citing the entry.
\item[short] Print the full citation the first time the entry is cited. Use
  the \bibfield{shorthand} on subsequent citations.
\item[intro] Print the full citation the first time the entry is cited
  followed by (henceforth cited as \bibfield{shorthand}). Use the
  \bibfield{shorthand} on subsequent citations.
\end{valuelist}


\end{optionlist}

\section{Database Guide}

\subsection{Entry Types}

All standard entry types of \biblatex{} are supported by \biblatexsbl. This
section gives an overview of entry types that are most relevant, unique to, or
treated in a custom way by \biblatexsbl{}.

\begin{typelist}

\typeitem{ancienttext}

This is a custom type for \biblatexsbl. It is used for the special examples in
\cite[§6.4.1, §6.4.3 and §6.4.8]{SBLHS}.

Unless \bibfield{options = \{skipbib=false\}} is set explicitly, an
\bibtype{ancienttext} entry will not appear in the bibliography. (Although,
see \opt{ANRW} \bibfield{entrysubtype} below for an exception.) The
\bibfield{related} field is used to refer to the entry which should appear in
the bibliography instead of the \bibtype{ancienttext} entry. Options can be
set on the related entry using the \bibfield{relatedoptions} field.

The entry pointed to by \bibfield{related} along with the \bibfield{postnote}
is printed in parentheses after the \bibfield{altpostnote}, \bibfield{editor},
and \bibfield{translator} fields if they are present. \bibfield{translator}
and \bibfield{editor} fields are omitted for subsequent citations. e.g.,

\begin{snugshade}
  \samplecite{1}[319]{suppiluliumas}
  \samplecite*{2}[319]{suppiluliumas}
  \samplebib{ANET}
\end{snugshade}

If the entry contains \bibfield{options = \{skipbib=false\}}, then the
bibliography entry will be like \bibtype{book}. Any shorthand is also
printed in the same way as a \bibtype{book} shorthand.

The following values for the \bibfield{entrysubtype} field are supported:

\begin{valuelist}

\item[ANRW]

The \opt{ANRW} \bibfield{entrysubtype} is particularly for citing \cite{ANRW}
as outlined in §6.4.8 of the \cite{SBLHS}. In this case, the entry \emph{will}
appear in the bibliography. See \sty{biblatex-sbl-examples.pdf} for full
details of the required database entry.

\item[chronicle]

Formats the \bibfield{title} using an upright shape font without quotation
marks. e.g.,

\begin{snugshade}
  \samplecite{1}[(lines 3--4)125]{esarhaddonchronicle}
\end{snugshade}

\item[COS]

Suppresses parentheses around \emph{COS} and the \bibfield{postnote} for
subsequent citations. e.g.,

\begin{snugshade}
  \samplecite{1}[44]{greathymnaten}
  \samplecite*{2}[44]{greathymnaten}
\end{snugshade}

\item[inscription]

  Similarly to \bibfield{entrysubtype = \{chronicle\}}, this formats the
  \bibfield{title} using an upright shape font without quotation marks.

\end{valuelist}

\typeitem{article}

An article in a journal or magazine. Also use this type for review articles
\parencite[§6.3.4]{SBLHS} and electronic journal articles
\parencite[§6.3.10]{SBLHS}.

\typeitem{book}

A single-volume book with one or more authors where the authors share credit
for the work as a whole.

\typeitem{inbook}

A part of a book which forms a self-contained unit with its own title.

\typeitem{bookinbook}

This type is similar to \bibtype{inbook} but intended for works originally
published as a stand-alone book. The main difference is that the title is
printed in italics instead of in quotation marks.

\typeitem{mvbook}

A multivolume \bibtype{book}.

There is one \bibfield{entrysubtype} supported:

\begin{valuelist}

\item[RIMA]

The citation for \citeseries{RIMA} \parencite[97]{SBLHS} is treated like a
series with a number when cited in full, but as a shorthand with a volume when
cited in short form. See \sty{biblatex-sbl-examples.pdf} for full details.

\end{valuelist}

\typeitem{suppbook}

Supplemental material in a \bibtype{book}. Use this for an introduction,
preface or foreword written by someone other than the author
\parencite[§6.2.14]{SBLHS}. The \bibfield{type} field is used to specify the
type of supplementary material. See §6.2.14 of
\sty{biblatex-sbl-examples.pdf}. If no \bibfield{type} is given, then this
behaves like an \bibtype{inbook}.

\typeitem{booklet}

A book-like work without a formal publisher or sponsoring institution.

\typeitem{classictext}

This type is a custom type for \biblatexsbl. It is used for the special
examples in \cite[§6.4.2 and §§6.4.4–6]{SBLHS}.

Unless \bibfield{options = \{skipbib=false\}} is set explicitly, a
\bibtype{classictext} entry will not appear in the bibliography. The
\bibfield{xref} field is used to refer to the entry which should appear in the
bibliography instead of the \bibtype{classictext} entry.

If present, the \bibfield{translator} and \bibfield{series} are printed in
parentheses following the \bibfield{postnote}. e.g.,

\begin{snugshade}
  \samplecite{1}[15.18-19]{tacitus:ann:jackson}
  \samplebib{tacitus}
\end{snugshade}

The \bibfield{series} can be suppressed by setting \bibfield{options =
\{useseries=false\}}.

If the entry contains \bibfield{options = \{skipbib=false\}}, then the
bibliography entry will be like \bibtype{incollection} except that the
\bibfield{title} is set in italics instead of within quotation marks.

The following values for the \bibfield{entrysubtype} field are supported:

\begin{valuelist}

\item[churchfather]

Entries using the \opt{churchfather} \bibfield{entrysubtype} print the entry
pointed to by \bibfield{related} within parentheses following the
\bibfield{altpostnote}. The \bibfield{postnote} field applies to the entry in
\bibfield{related}. \bibfield{relatedoptions} can be used to control some
aspects of the formatting for the related entry. \bibfield{altpostnote} is
always separated from the title by a space.

\begin{snugshade}
  \samplecite{1}[(28.3.5)252]{augustine:letters}
  \samplebib{augustine:letters}
\end{snugshade}

\end{valuelist}

\typeitem{collection}

A single-volume collection with multiple, self-contained contributions by
distinct authors which have their own title. The work as a whole has no
overall author but it will usually have an editor.

\typeitem{mvcollection}

A multi-volume \bibtype{collection}.

\typeitem{incollection}

A contribution to a collection which forms a self-contained unit with a
distinct author and title.

\typeitem{commentary}

A single-volume commentary on a book (or part of a book) of the Bible by one
or more authors. This entry type is similar to \bibtype{book}, except that any
\bibfield{volume} and \bibfield{maintitle} is only printed in the
bibliography, not the citation.

\typeitem{mvcommentary}

A multi-volume commentary on a single book of the Bible by one or more authors
or a multi-volume commentary on the whole Bible by multiple authors. Unlike
\bibtype{commentary}, this behaves exactly the same as a \bibtype{mvbook}.

\typeitem{incommentary}

A contribution to a commentary which forms a self-contained unit with a
distinct author and title. This is typically a commentary on a book of the
Bible appearing in a single or multi-volume commentary on the entire Bible.

If an entry contains an \bibfield{xref} field, then the bibliography entry is
printed in either short or long form as described above under
\opt{fullbibrefs}.

\typeitem{conferencepaper}

An unpublished paper presented at a professional society. Use the
\bibfield{eventtitle}, \bibfield{venue}, and \bibfield{date} fields to
specify detail for the conference. See §6.3.8 of
\sty{biblatex-sbl-examples.pdf} for and example.

\typeitem{lexicon}

A single-volume lexicon or theological dictionary. This is similar to a
\bibtype{book}.

\typeitem{mvlexicon}

A multi-volume lexicon or theological dictionary. This is similar to a
\bibtype{mvbook}.

\typeitem{inlexicon}

An article in a lexicon or theological dictionary. This is a custom type for
\biblatexsbl. The required \bibfield{xref} field must contain the entry name
of a \bibtype{lexicon} or \bibtype{mvlexicon}. The \bibtype{inlexicon} entry
does not appear in the bibliography. Instead the lexicon pointed to by
\bibfield{xref} appears in the bibliography.

Subsequent citations do not include the \bibfield{title}, only the name of the
lexicon (specified by the \bibfield{xref} entry). e.g.,

\begin{snugshade}
  \samplecite{1}[511]{dahn+liefeld:see+vision+eye}
  \samplecite{2}[511]{dahn+liefeld:see+vision+eye}
  \samplebib{NIDNTT}
\end{snugshade}

\typeitem{misc}

A fallback type for entries which do not fit into any other category. Use the
\bibfield{howpublished} field to supply publishing information in free format,
if applicable.

\typeitem{online}

An online resource without a print counterpart. This is similar to an
\bibtype{article}.

\typeitem{periodical}

A complete issue of a periodical, such as a special issue of a journal. The
title of the periodical is given in the \bibfield{title} field. If the issue
has its own title in addition to the main title of the periodical, it goes in
the \bibfield{issuetitle} field.

This type could also be used to insert a journal into the list of
abbreviations. In this case, just use \bibfield{title}, \bibfield{shorttitle},
and set \bibfield{options = \{skipbib\}}.

\typeitem{reference}

A single-volume encyclopaedia or dictionary. This is similar to a
\bibtype{book}.

\typeitem{mvreference}

A multi-volume \bibtype{reference}. This is similar to a \bibtype{mvbook}.

\typeitem{inreference}

An article in an encyclopaedia or dictionary. The required \bibfield{xref}
field must contain the entry name of a \bibtype{reference} or
\bibtype{mvreference}.

The bibliography entry is printed in either short or long form as described
above under \opt{fullbibrefs}.

\typeitem{review}

A book review in a journal. This is similar to an \bibtype{article}. Use the
\bibfield{revdauthor}\slash\bibfield{revdeditor} and \bibfield{revdtitle}
fields to specify the author\slash editor and title of the book being
reviewed.

Note that review articles are treated like articles and should use the
\bibtype{article} entry type.

\typeitem{seminarpaper}

An \citeshorthand{SBL} seminar paper. This is similar to an
\bibtype{incollection} except that \emph{in} is suppressed byfore the
\bibfield{booktitle}\slash\bibfield{maintitle}. See §6.4.11 of
\sty{biblatex-sbl-examples.pdf} for an example.

\typeitem{series}

A multi-volume series. This is similar to a \bibtype{mvbook} except that an
upright shaped font is used for the \bibfield{title} field and the
\bibfield{option} field is set to \bibfield{\{useauthor=false,}
\bibfield{useditor=false\}} by default.

This type could also be used to place a series in the list of abbreviations.
In this case, just use \bibfield{series}, \bibfield{shortseries}, and set
\bibfield{options = \{skipbib\}}.

\end{typelist}

\subsection{Entry Fields}

\biblatexsbl supports all entry fields from the \biblatex manual except for
\bibfield{pagetotal}. There are also a number of custom entry fields and
specially handled fields supported by \biblatexsbl. These are documented
below.

\begin{fieldlist}

\listitem{bookeditor}{name}

The editor(s) of the \bibfield{booktitle}.

The behaviour of \bibfield{editor}, \bibfield{bookeditor}, and
\bibfield{maineditor} is as follows (when \bibfield{editor} is not used as
the overall editor): \bibfield{editor} applies to \bibfield{maintitle} (if
set) unless \bibfield{maineditor} is set. In this case, \bibfield{editor}
applies to \bibfield{booktitle} (if set) unless \bibfield{bookeditor} is
set. In this case, \bibfield{editor} applies to \bibfield{title}.

\listitem{booktranslator}{name}

The translator(s) of the \bibfield{booktitle}.

The behaviour of \bibfield{translator}, \bibfield{booktranslator}, and
\bibfield{maintranslator} is the same as for \bibfield{editor},
\bibfield{bookeditor}, and \bibfield{maineditor}.

\fielditem{eprintdate}{date}

The date a text edition published online with no print counterpart or an
article in an online database is released. See §6.4.1 and §6.4.13 of
\sty{biblatex-sbl-examples.pdf}.

\fielditem{eprintday}{datepart}

This field holds the day component of the \bibfield{eprintdate} field.

\fielditem{eprintmonth}{datepart}

This field holds the month component of the \bibfield{eprintdate} field.

\fielditem{eprintyear}{datepart}

This field holds the year component of the \bibfield{eprintdate} field.

\listitem{maineditor}{name}

The editor(s) of the \bibfield{maintitle}.

\listitem{maintranslator}{name}

The translator(s) of the \bibfield{maintitle}.

\listitem{revdauthor}{name}

The author(s) of the \bibfield{revdtitle}.

\listitem{revdeditor}{name}

The editor(s) of the \bibfield{revdtitle}.

\fielditem{revdshorttitle}{literal}

The title of a book being review in an abridged form. This field is used in
subsequent citations of \bibtype{review} entry types.

\fielditem{revdsubtitle}{literal}

The subtitle of a book being reviewed.

\fielditem{revdtitle}{literal}

The title of a book being reviewed.

\fielditem{revdtitleaddon}{literal}

An annex to the \bibfield{revdtitle}, to be printed in a different font.

\fielditem{seriesseries}{literal}

This field is used when a \bibfield{series} is begun anew to distinguish
between the old and new series. See \cite[§6.2.24]{SBLHS}.

\fielditem{shortbooktitle}{literal}

The \bibfield{booktitle} in abridged form.

\fielditem{shorthand}{literal}

A special short form printed instead of the usual citation. Sometimes the
\bibfield{shorthand} is short for the authors and should be printed in an
upright font shape. Other times it is short for the \bibfield{title} and
should be printed in italics. This behaviour is controlled using the
\bibfield{shorttitle} field. If the \bibfield{shorttitle} is the same as the
\bibfield{shorthand} then the \bibfield{shorthand} is assumed to be short for
the \bibfield{title} and printed in italics, otherwise it is printed in an
upright font. The \bibfield{shorthand} is automatically inserted into the list
of abbreviations.

The separator between the \bibfield{shorthand} and \bibfield{postnote} depends
on the content of the \bibfield{postnote}. If the \bibfield{postnote} contains
a ‘.’, ‘:’, or ‘§’, then the separator is a space, otherwise it is a comma.

\fielditem{shortjournal}{literal}

The \bibfield{journaltitle} in abridged form. This is always printed instead
of the \bibfield{journaltitle}. The \bibfield{shortjournal} and
\bibfield{journaltitle} is then automatically inserted into the list of
abbreviations.

\fielditem{shortmaintitle}{literal}

The \bibfield{maintitle} in abridged form.

\fielditem{shortseries}{literal}

The \bibfield{series} in abridged form. This is always printed instead of the
\bibfield{series}. The \bibfield{shortseries} and \bibfield{series} is then
automatically inserted into the list of abbreviations.

\fielditem{shorttitle}{literal}

The \bibfield{title} in abridged form. This is printed instead of the full
title on subsequent citations.

\listitem{withauthor}{name}

The author(s) who assist the \bibfield{author}. See \bibfield{witheditortype},
below, for an example.

\fielditem{withauthortype}{literal}

The type of \bibfield{withauthor}. This field will affect the string used to
introduce the author(s) who assist the author. If unspecified, the
bibliography string \sty{with} is used.

\listitem{witheditor}{name}

The editor(s) who assist the \bibfield{editor}.

\fielditem{witheditortype}{literal}

The type of \bibfield{witheditor}. This field will affect the string used to
introduce the editor(s) who assist the editor. If unspecified, the
bibliography string \sty{with} is used.

\begin{snugshade}
  \samplecite{1}[1:24]{TLOT}
  \samplebib{TLOT}
  \samplebiblist{TLOT}
\end{snugshade}

\listitem{withtranslator}{name}

The translator(s) who assist the \bibfield{translator}.

\fielditem{withtranslatortype}{literal}

The type of \bibfield{withtranslator}. This field will affect the string used
to introduce the translator(s) who assist the translator. If unspecified, the
bibliography string \sty{with} is used.

\listitem{withbookauthor}{name}

The author(s) who assist the \bibfield{bookauthor}.

\fielditem{withbookauthortype}{literal}

This field is analogous to the \bibfield{withauthortype}, but for the
\bibfield{bookauthor}.

\listitem{withbookeditor}{name}

The editor(s) who assist the \bibfield{bookeditor}.

\fielditem{withbookeditortype}{literal}

This field is analogous to the \bibfield{witheditortype}, but for the
\bibfield{bookeditor}.

\listitem{withbooktranslator}{name}

The translator(s) who assist the \bibfield{booktranslator}.

\fielditem{withbooktranslatortype}{literal}

This field is analogous to the \bibfield{withtranslatortype}, but for the
\bibfield{booktranslator}.

\listitem{withmainauthor}{name}

The author(s) who assist the \bibfield{mainauthor}.

\fielditem{withmainauthortype}{literal}

This field is analogous to the \bibfield{withauthortype}, but for the
\bibfield{mainauthor}.

\listitem{withmaineditor}{name}

The editor(s) who assist the \bibfield{maineditor}.

\fielditem{withmaineditortype}{literal}

This field is analogous to the \bibfield{witheditortype}, but for the
\bibfield{maineditor}.

\listitem{withmaintranslator}{name}

The translator(s) who assist the \bibfield{maintranslator}.

\fielditem{withmaintranslatortype}{literal}

This field is analogous to the \bibfield{withtranslatortype}, but for the
\bibfield{maintranslator}.

\end{fieldlist}

\subsection{Type and Entry Options}

\biblatexsbl supports many of the entry options outlined in the \biblatex
manual. There are also a number of custom entry options supported by
\biblatexsbl. These are documented below.

\begin{optionlist}

\optitem[false]{skipbiblistshorthand}{\opt{true}, \opt{false}}

This option controls what appears in the list of abbreviations for database
entries containing both a \bibfield{shorthand} and a \bibfield{shortseries}.
For entries not containing a \bibfield{shortseries} just use the option
\opt{skipbiblist}. The possible options are:

\begin{valuelist}
\item[true] Do not include the \bibfield{shorthand} in the list of
  abbreviations.
\item[false] Include the \bibfield{shorthand} in the list of abbreviations.
\end{valuelist}

\optitem[false]{skipbiblistshortseries}{\opt{true}, \opt{false}}

This option controls what appears in the list of abbreviations for database
entries containing both a \bibfield{shorthand} and a \bibfield{shortseries}.
For entries not containing a \bibfield{shorthand} just use the option
\opt{skipbiblist}. The possible options are:

\begin{valuelist}
\item[true] Do not include the \bibfield{shortseries} in the list of
  abbreviations.
\item[false] Include the \bibfield{shortseries} in the list of abbreviations.
\end{valuelist}

\optitem[true]{usefullcite}{\opt{true}, \opt{false}}

This options controls the format of first citations. The possible choices are:

\begin{valuelist}
\item[true] Use a full citation the first time the entry is cited.
\item[false] Use the short citation form the first time the entry is cited.
\end{valuelist}

\optitem[true]{useseries}{\opt{true}, \opt{false}}

This option controls whether the \bibfield{series} is printed in parentheses
following a \bibtype{classictext} citation. This does not affect other entry
types.

\begin{valuelist}
\item[true] Print the \bibfield{series}.
  \begin{snugshade}
    \samplecite{1}[2.233-235]{josephus:ant:thackery}
    \samplebib{josephus}
  \end{snugshade}
\item[false] Suppress printing the \bibfield{series}.
  \begin{snugshade}
    \samplecite{2}[10]{heraclitus:epistle1:worley}
    \samplebib{heraclitus:epistle1:worley}
  \end{snugshade}
\end{valuelist}

\optitem[true]{useshorttitle}{\opt{true}, \opt{false}}

This option controls the format of subsequent citations. The possible choices
are:

\begin{valuelist}
\item[true] Include the \bibfield{shorttitle} or \bibfield{title} in
  subsequent citations.
\item[false] Suppress the \bibfield{shorttitle} or \bibfield{title} in
  subsequent citations, so only the author(s) or editor(s) are printed.
\end{valuelist}

\optitem[true]{usevolume}{\opt{true}, \opt{false}}

This option controls whether the \bibfield{volume} is printed as part of the
citation text or as part of the \bibfield{postnote}.

\begin{valuelist}
\item[true] Print the \bibfield{volume} as part of the main citation
  information. e.g., “Vol.~1.”
\item[false] Print the \bibfield{volume} field as part of the
  \bibfield{postnote}. e.g., “1:”
\end{valuelist}

\end{optionlist}

\subsection{Reprints}

\biblatexsbl supports three different ways of doing reprints with varying
complexity.

If only the original publisher, location, and/or year are required, then use
the fields \bibfield{origpublisher}, \bibfield{origlocation}, and
\bibfield{origdate}. e.g.,

\begin{quote}
\begin{lstlisting}{}
@book{vanseters:1997,
    author = {Van Seters, John},
    title = {In Search of History: Histeriography in the Ancient
               World and the Origins of Biblical History},
    origlocation = {New Haven},
    origpublisher = {Yale University Press},
    origdate = {1983},
    location = {Winona Lake, IN},
    publisher = {Eisenbrauns},
    date = {1997}
}
\end{lstlisting}
\begin{snugshade}
  \samplecite{1}[90]{vanseters:1997}
  \samplebib{vanseters:1997}
\end{snugshade}
\end{quote}

When extra information is required, use a related entry with
\bibfield{relatedtype = \{reprint\}}. A custom string can be specified instead
of “Repr.” using the optional \bibfield{relatedstring} field. In this case no
punctuation is inserted after the \bibfield{relatedstring}. You could think of
the default being \bibfield{relatedstring = \{\cmd{bibstring}\{reprint\},\}}.
e.g.,

\begin{quote}
\begin{lstlisting}{}
@mvbook{sasson:2000,
    editor = {Sasson, Jack M.},
    title = {Civilizations of the Ancient Near East},
    volumes = {4},
    location = {New York},
    publisher = {Scribner's Sons},
    year = {1995},
    related = {sasson:repr},
    relatedtype = {reprint}
}
@mvbook{sasson:repr,
    volumes = {4~vols.\ in 2},
    location = {Peabody, MA},
    publisher = {Hendrickson},
    date = {2000}
}
\end{lstlisting}
\begin{snugshade}
  \samplecite{1}[1:40]{sasson:2000}
  \samplebib{sasson:2000}
\end{snugshade}
\end{quote}

A full reprint history also uses the \bibfield{related} field, but with some
other \bibfield{relatedtype} apart from \bibfield{relatedtype = \{reprint\}}.
e.g.,

\begin{quote}
\begin{lstlisting}{}
@book{wellhausen:1883,
    author = {Wellhausen, Julius},
    title = {Prolegomena zur Geschichte Israels},
    edition = {2},
    location = {Berlin},
    publisher = {Reimer},
    date = {1883}
}
@book{wellhausen:1885,
    author = {Wellhausen, Julius},
    title = {Prolegomena to the History of Israel},
    translator = {Black, J. Sutherland and Enzies, A.},
    preface = {Smith, W. Robertson},
    location = {Edinburgh},
    publisher = {Black},
    related = {wellhausen:1883},
    relatedtype = {translationof},
    date = {1885}
}
@book{wellhausen:1957,
    author = {Wellhausen, Julius},
    title = {Prolegomena to the History of Ancient Israel},
    location = {New York},
    publisher = {Meridian Books},
    related = {wellhausen:1885},
    relatedtype = {reprintof},
    date = {1957}
}
\end{lstlisting}
\begin{snugshade}
  \samplecite{1}[20]{wellhausen:1957}
  \samplebib{wellhausen:1957}
\end{snugshade}
\end{quote}

\section{Important Changes}

\subsection*{0.9\quad 2018-05-20}

\begin{itemize}[itemsep=0pt]
  \item Support and require \biblatex 3.11.
  \item Use ibid and idem by default.
  \item \sty{polyglossia} is no longer supported. You should use \sty{babel}
    instead.
\end{itemize}

\subsection*{0.8.2\quad 2017-11-16}

\begin{itemize}[itemsep=0pt]
  \item Support and require \biblatex 3.8a
\end{itemize}

\printbiblist[heading=biblistintoc]{abbreviations}

\printbibliography[heading=bibintoc]

\printindex

\end{document}
